%%%%%%%%%%%%%%%%%%%%%%%%%%%%%%%%%%%%%%%
% Author : 
% Mit Naria
% 
% Github repo : 
% https://github.com/foxtrot9/onepage-resume (Might take some time before I push changes.)
%
% Last updated: 22nd Feb, 2018.
% 
% Added features:
% * Multiline title heading using \texorpdfstring and \newline command
% * Added Fontawesome emoticons
% * Structure changed to suite needs of recent graduates and Indian undergraduate students
% * Clean and polished
% Known issues:
% * Same as mentioned below
%%%%%%%%%%%%%%%%%%%%%%%%%%%%%%%%%%%%%%%

%%%%%%%%%%%% INSPIRED FROM: %%%%%%%%%%%

%%%%%%%%%%%%%%%%%%%%%%%%%%%%%%%%%%%%%%%
% Deedy - One Page Two Column Resume
% LaTeX Template
% Version 1.1 (30/4/2014)
%
% Original author:
% Debarghya Das (http://debarghyadas.com)
%
% Original repository:
% https://github.com/deedydas/Deedy-Resume
%
% IMPORTANT: THIS TEMPLATE NEEDS TO BE COMPILED WITH XeLaTeX
%
% This template uses several fonts not included with Windows/Linux by
% default. If you get compilation errors saying a font is missing, find the line
% on which the font is used and either change it to a font included with your
% operating system or comment the line out to use the default font.
% 
%%%%%%%%%%%%%%%%%%%%%%%%%%%%%%%%%%%%%%
% 
% TODO:
% 1. Integrate biber/bibtex for article citation under publications.
% 2. Figure out a smoother way for the document to flow onto the next page.
% 3. Add styling information for a "Projects/Hacks" section.
% 4. Add location/address information
% 5. Merge OpenFont and MacFonts as a single sty with options.
% 
%%%%%%%%%%%%%%%%%%%%%%%%%%%%%%%%%%%%%%
%
% CHANGELOG:
% v1.1:
% 1. Fixed several compilation bugs with \renewcommand
% 2. Got Open-source fonts (Windows/Linux support)
% 3. Added Last Updated
% 4. Move Title styling into .sty
% 5. Commented .sty file.
%
%%%%%%%%%%%%%%%%%%%%%%%%%%%%%%%%%%%%%%%
%
% Known Issues:
% 1. Overflows onto second page if any column's contents are more than the
% vertical limit
% 2. Hacky space on the first bullet point on the second column.
%
%%%%%%%%%%%%%%%%%%%%%%%%%%%%%%%%%%%%%%


\documentclass[]{deedy-resume-openfont}


\begin{document}


%%%%%%%%%%%%%%%%%%%%%%%%%%%%%%%%%%%%%%
% NEW COMMANDS
%%%%%%%%%%%%%%%%%%%%%%%%%%%%%%%%%%%%%%
\newcommand{\mybullet}{\tiny{\bullet}}

%%%%%%%%%%%%%%%%%%%%%%%%%%%%%%%%%%%%%%
%
%     LAST UPDATED DATE
%
%%%%%%%%%%%%%%%%%%%%%%%%%%%%%%%%%%%%%%
\lastupdated

%%%%%%%%%%%%%%%%%%%%%%%%%%%%%%%%%%%%%%
%
%     TITLE NAME
%
%%%%%%%%%%%%%%%%%%%%%%%%%%%%%%%%%%%%%%


\namesection{}{Mit Naria}{ % { \urlstyle{same}\url{http://mitnaria.com} \\
\hspace*{10pt} \href{https://github.com/foxtrot9}{\faGithub \hspace*{1pt} foxtrot9} \hspace*{3pt} |
\hspace*{3pt} \href{mailto:mit4dev@gmail.com}{\faEnvelope \hspace*{1pt} mit4dev@gmail.com} \hspace*{3pt} | 
\hspace*{1pt} \href{tel:+353-873697915}{\faPhone \hspace*{1pt} +353-873697915} |
\hspace*{3pt} \href{https://www.linkedin.com/in/Mit-Naria}{\faLinkedinSign \hspace*{1pt} Mit-Naria}
}

%%%%%%%%%%%%%%%%%%%%%%%%%%%%%%%%%%%%%%
%
%     COLUMN ONE
%
%%%%%%%%%%%%%%%%%%%%%%%%%%%%%%%%%%%%%%

\begin{minipage}[t]{0.33\textwidth} 

%%%%%%%%%%%%%%%%%%%%%%%%%%%%%%%%%%%%%%
%     EDUCATION
%%%%%%%%%%%%%%%%%%%%%%%%%%%%%%%%%%%%%%

\section{Education} 

\subsection{\texorpdfstring{Dhirubhai Ambani \newline Institute of Information \newline and Communication \newline Technology (DAIICT)}{}}
\descript{B.Tech in ICT with minor in Computational Science}
\location{Gandhinagar, India\\ 
Apr, 2018 \\ 
Cum. GPA: 8.03 / 10}
\sectionsep

%%%%%%%%%%%%%%%%%%%%%%%%%%%%%%%%%%%%%%
%     LINKS
%%%%%%%%%%%%%%%%%%%%%%%%%%%%%%%%%%%%%%

% \section{Links} 
% \faGithub \hspace*{1pt} Github:// \href{https://github.com/foxtrot9}{\custombold{foxtrot9}} \\
% \faLinkedinSign \hspace*{1pt} LinkedIn://  \href{https://www.linkedin.com/in/Mit-Naria}{\custombold{Mit-Naria}} \\
% Quora://  \href{https://www.quora.com/profile/Mit-Naria}{\custombold{Mit-Naria}} \\
% \faTwitter \hspace*{1pt} Twitter:// \href{https://twitter.com/mit_naria}{\custombold{mit\_naria}}
% \faStackOverflow \hspace*{1pt} StackOverflow:// \href{https://stackoverflow.com/users/6518787/foxtrot9}{\custombold{foxtrot9}}
% \sectionsep


%%%%%%%%%%%%%%%%%%%%%%%%%%%%%%%%%%%%%%
%     SKILLS
%%%%%%%%%%%%%%%%%%%%%%%%%%%%%%%%%%%%%%

\section{Skills}
\subsection{Programming}
\location{Intermediate:}
\textbullet{} Java  \textbullet{} C \\
\location{Begineer:}
\textbullet{} Matlab \textbullet{} Assembly \\ %\textbullet{} \LaTeX{}\\
\location{Familiar:}
\textbullet{} Python \textbullet{} SQL \textbullet{} C++ \\
\textbullet{} R \textbullet{} Shell \textbullet{} Lua
\sectionsep

\subsection{Tools}
\textbullet{} Kubernetes \textbullet{} Docker Swarm

\textbullet{} Terraform \textbullet{} Puppet \textbullet{} Ansible

\textbullet{} Nginx \textbullet{} Apache

\textbullet{} Redis \textbullet{} MySQL

\textbullet{} Elasticsearch \textbullet{} Rsyslog \textbullet{} Logstash

\textbullet{} Netdata \textbullet{} Prometheus \textbullet{} Icinga

\textbullet{} Grafana \textbullet{} Kibana \textbullet{} Statsd

\textbullet{} Jenkins \textbullet{} Bitbucket
\sectionsep

\subsection{Cloud platforms}
\textbullet{} AWS \textbullet{} Heroku
\sectionsep

\subsection{Libraries}
\textbullet{} OpenMP \textbullet{} MPI
\sectionsep

%%%%%%%%%%%%%%%%%%%%%%%%%%%%%%%%%%%%%%
%     COURSEWORK
%%%%%%%%%%%%%%%%%%%%%%%%%%%%%%%%%%%%%%

\section{Coursework}
\subsection{Undergraduate}
Operating Systems \\
System and Network Security \\
Database Management Systems \\
Computer Networks \\
Security Protocols \\
High Performance Computing\\
Software Engineering\\
Automata Theory\\
Data Structures and Algorithms\\
System Softwares\\
Data Analysis and Visualization\\
Object Oriented Programming\\
\sectionsep

%%%%%%%%%%%%%%%%%%%%%%%%%%%%%%%%%%%%%%
%
%     COLUMN TWO
%
%%%%%%%%%%%%%%%%%%%%%%%%%%%%%%%%%%%%%%

\end{minipage} 
\hfill
\begin{minipage}[t]{0.66\textwidth}

%%%%%%%%%%%%%%%%%%%%%%%%%%%%%%%%%%%%%%
%     EXPERIENCE
%%%%%%%%%%%%%%%%%%%%%%%%%%%%%%%%%%%%%%

\section{Experience}

\runsubsection{\href{https://www.google.com}{Google}}
\descript{| Software Engineer, Site Reliability Engineering}
\location{ July 2020 - Present}

Currently part of Google Search SRE team and working on Assistant and Autocomplete products.

\textbullet{} Worked on improving the reliability of the Assistant microservices by creating software for safely allowing replay of production traffic. It improved the coverage of pre-production testing and resulted in reduced production bugs and outages.

\textbullet{} Worked on improving reliability of the product by implementing canary best practices.

\textbullet{} Worked on managing 10s of microservices by building standard templates for various phases of software lifecycles. Added unit-tested code for alerting and monitoring.

\textbullet{} Worked on improving the horizontal scalability of probers by migrating it to shared infrastructure and adding functionality the probing system. It resolved horizontal scaling issues and resulted in reduced oncall toil and maintenance load.

\sectionsep

% Continuous Integration/Deployment of Applications and Cron Management.
% Log Collection System using Elasticsearch, Logstash and Kibana.
% Live system monitoring and alerting to ensure availability, reliability and fault tolerant running of services.
% Automation of tasks through scripts in Shell/Python.

% Responsibilities
% * Running and managing applications on dedicated servers.
% * Continuous Integration/Deployment of Applications and Cron Management.
% * Log Collection System using Elasticsearch, Logstash and Kibana.
% * Live system monitoring and alerting to ensure availability, reliability and fault tolerant running of services.
% * Automation of tasks through scripts in Shell/Python.

\runsubsection{\href{http://media.net}{Directi - Media.net}}
\descript{| Site Reliability Engineer}
\location{ Jun 2018 - July 2020}
I worked in Apps team and Email Monetization team of Media.net where I was responsible for designing, implementing and building high-availability scalable production systems.

\textbullet{} Worked on designing highly-available autoscaling systems using Kubernetes
- resulting in improved reliability, increased availability and 75\% reduction in infrastructure costs.

\textbullet{} Worked on AWS services such as EC2, EC2 Autoscaling, EKS, RDS, ALB/ELB, Fargate, Cloudwatch, 
S3, Redshift, VPC, SQS and Elasticache.

\sectionsep

\textbullet{} Designed distributed, scalable and real-time application metrics collection 
system using Statsd and Netdata.

\textbullet{} Developed log collection and visualization system using Elasticsearch, Rsyslog and Kibana -
removing single point of failure of existing system and achieving high throughput.

\textbullet{} Worked on continuous integration and deployment, automated software provisioning, 
configuration management for team infrastructure using Ansible, Terraform, Puppet Jenkins and Bitbucket pipelines.


% \textbullet{} I have Set up ELK stack and made all critical systems like MySQL, Redis to be highly available.
% \textbullet{} I have experience with working on Infrastructure-As-Code (IAC) process to maintain and upgrade infrastructure using Puppet, Ansible and Terraform).
%Have experience with working on AWS managed tools - EC2, EC2 Auto scaling, ECS, EKS, ELB, ALB, RDS, Elasticache, IAM, Redshift and Heroku platform.
%Experience with building, optimizing CI/CD pipelines and using tools such as Jenkins, Bitbucket pipelines.
%Experience with operating MySQL and Elasticsearch cluster.
%Experience with working on monitoring tools like Prometheus, Icinga and visualization tools like Grafana.
\sectionsep

%%%%%%%%%%%%%%%%%%%%%%%%%%%%%%%%%%%%%%
%     PROJECTS
%%%%%%%%%%%%%%%%%%%%%%%%%%%%%%%%%%%%%%

\section{Projects}

\subsection{Jenkins \hspace*{1pt} | \hspace*{1pt} Open Source Contribution}
\location{Feb 2019 - May 2019 \hspace*{1pt} | \hspace*{1pt} \href{https://github.com/jenkinsci/jenkins/pull/3915}{\faCodeFork}}
\textbullet{} Worked on adding a feature to support excluding patterns in fingerprint module in Jenkins and fixed a bug. Tech: Java
\sectionsep

\subsection{Sample Sort}
\location{Oct 2016 - Nov 2016 \hspace*{1pt} | \hspace*{1pt} \href{https://github.com/foxtrot9/HPC_Project}{\faCodeFork} }
\textbullet{} Implemented a parallel sorting algorithm using MPI protocol which sorted 2 billion
64-bit integers in just 105 seconds on Intel Xeon E5-2650. Tech: C, OpenMP, MPI
\sectionsep

%\subsection{Completely Fair Scheduler}
%\location{Mar 2017}
%CFS is default scheduling algorithm in Linux kernel v2.6.23+. I implemented it as part of
%Operating System course at DAIICT. Tech: C
%\sectionsep

% \subsection{Dynamic Memory Allocator}
% \location{Mar 2016}
% Implemented a custom memory allocator similar to malloc using C and Linux syscalls. %as part of System Software course at DAIICT. Tech: C, Linux
% \sectionsep

%%%%%%%%%%%%%%%%%%%%%%%
%     OLD PROJECTS
%%%%%%%%%%%%%%%%%%%%%%%
%
% \subsection{Tiny Shell}
% \location{Feb 2016}
% Built a tiny shell with features such as job control and I/O redirection.
% \sectionsep
%


%%%%%%%%%%%%%%%%%%%%%%%%%%%%%%%%%%%%%%
%     AWARDS
%%%%%%%%%%%%%%%%%%%%%%%%%%%%%%%%%%%%%%

% \section{Awards} 
% \begin{tabular}{rl}
%2016        & HiPC Student Parallel Programming challenge phase 1 finalists.\\
% 2014-18		& Recipient of Merit-Cum-Means scholarship awarded by DAIICT. \\
% 2014        & Rank-wise 6 in DAIICT according to AIR. \\
% 2014        & AIR 11520 in JEE (Advanced) 2014. \\	
% 2014        & Selected for CM scholarship awarded by state of Gujarat. \\
% 2014        & AIR 2162 in JEE (Main) 2014.
% \end{tabular}
% \sectionsep


\end{minipage} 
\end{document}
